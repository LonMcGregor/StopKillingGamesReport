\chapter{Action}

\section{Recommendations}
\lm{Include some recommendations for action or how laws / instructions for consumer body groups could work}
These are a set of recommendations that could be put in place to achieve the goals of the Stop Killing Games campaign.


\subsection{Recommendations for publishers}

The recommendations would apply to any game where:
\begin{itemize}
    \item A player has purchased the game
    \item A player has made an in-app purchase in a free-to-play game
    \item The publisher has decided to end support for the game
\end{itemize}

If the game has a singleplayer component:
\begin{itemize}
    \item The core singleplayer part of the game must be playable
    \item Non-essential components such as leaderboards or other online decorations do not need to be supported
\end{itemize}

If the game relies on \gls{streamedassets}:
\begin{itemize}
    \item The publisher must offer an \gls{asset} containing at least the minimum viable resources needed to be able to play the game without support
    \item The publisher should offer documentation allowing for players to download or make their own copies of online \glspl{asset}
\end{itemize}

If the game has a multiplayer component:
\begin{itemize}
    \item Local play must be playable
    \item Server based play must be playable through a server software made available by the publisher
    \item The publisher should offer documentation allowing for setting up and managing server discovery, \gls{matchmaking}
\end{itemize}

If the game has DRM, Anti-Cheat, or some other technical measure that relies on an online connection:
\begin{itemize}
    \item A game must be updated to remove it at end of support
    \item Or it must be changed so that an online connection is not required for it to function after end of support
\end{itemize}

If there is a provisioned support period:
\begin{itemize}
    \item The publisher could notify the consumer prior to purchase how long that support period will last
\end{itemize}


\subsection{Recommendations for government}

Create a consumer action group or ombudsman capable of investigating claims of game support loss and subsequent loss of play,
or grant powers to an existing consumer action group capable of offering this support.

Ensure that when someone purchases a game, this is considered the purchase of a good, and all the rights that consumers are subsequently entitled to apply.
