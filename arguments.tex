\chapter{Arguments}

\section{Arguments in favour of our proposals}
In this section we will summarise the main arguments we are making for why we want new regulations to stop publishers killing games.

\subsection{Player Choice}
Players want to play older games.
A recent report found that at up to 60\% of play time is spent playing games that are 6 years old or older~\cite{kotaku-oldgames-2024}.
This shows that recency or newness is not necessarily a major factor players consider when choosing what game to play.
However, when a publisher takes steps that result in killing a game, that takes choice away from players.

\lm{How to make the argument that publishers shouldn't be allowed to kill older games to push players to sequels?
Example the Overwatch 1 > 2 debacle.
This isn't anti-competitive in the strictest sense as it's all within one publisher.}

\subsection{Ownership}
When a consumer purchases a physical product, they own that product.
After point of sale, neither the manufacturer nor the seller of the product can take interact with that owned product without the purchasing consumer's consent.
They can neither take it back nor they cannot modify it.
If there is some reason to do this, they can issue a recall or offer a refund, and the choice is left to the consumer whether to take up this offer.

With digital only and internet connected products, this paradigm is changed dramatically.
Unless the consumer takes specific action to protect their purchase\footnote{
Owners may resort to making personal backup copies or may take measures to remove DRM,
but depending on interpretation of the law this may or may not be legal.
},
manufacturers and distributors of digital products have a strong level of control which takes precedence over the ownership rights of the consumer.
Choices publishers make could result in unwanted changes to the products or damage to the products rendering them unusable as evidenced in this report,
effectively taking purchased products away from the consumers who bought them.

During a support period, where prior to purchase consumers are made aware that publisher led changes may take place, this kind of activity is not in contention.
However, after the support period ends, we argue that selling products which will end up unusable, preventing the consumers who purchased them to retain them in the long term, is tantamount to theft.
\lm{Is theft the correct term to use here?}

\subsection{Planned Obsolescence}
All physical goods have a shelf life, either because the way they are manufactured allows them to degrade, or as a result of natural wear and tear.
With digital goods there is no wear and tear, so when the support period ends, there is no reason that a digital file (such as a videogame) should have to stop functioning.

Some physical goods are sold with a warranty for a set period disclosed at time of purchase, in which manufacturing faults may be repaired at no cost.
An analogy for videogames is the support period during which the publisher will make a game available in its entirety.
When the warranty (support) for a physical good expires, the item that was under warranty does not immediately stop functioning.
When a support period ends, there is no practical reason why a consumer should not be allowed to continue using the product they purchased,
as long as the customer retains a working copy of the game \glspl{asset} and system it was intended for.

The publisher would not be obligated to provide any support for things like malfunction or bugs after the support period ends,
much as a physical item sold under warranty does not carry any support requirement after the warranty expires.
A corollary is that publishers of videogames should not be allowed remove functionality at the end of support.
For them to do so is effectively a form of planned obsolescence.
\lm{Is this what it is?}

\subsection{Consumer Confidence}
Consumer confidence is related to planned obsolescence.
Videogame players are increasingly aware of the risk of online only games,
and players say this may factor in their decision to purchase games\cite{2k-topspin-2024}.
A survey found that gamers are wary of buying associated hardware that is online-only,
citing ``wanting to play their current physical games in the future'' as a factor\cite{console-wary-2019}.
Having a framework in which consumers can be confident about the support period for their purchase, and a plan of what will happen at the end of this,
could result in increased sales for publishers and the industry as a whole.

In practice, many publishers will release ``roadmaps'' discussing release dates for new game, planned new content such as \gls{dlc}, and different competitions.
Unfortunately, these roadmaps often omit an end date and a description of what state the game will be left in when support ends.
This means that potential buyers are missing key information that may be relevant to their purchase, if game longevity is a factor they are worried about.
\lm{Are there \emph{any} games that explained their end of support date at launch time?
Has any publisher ever bothered to plan this out?}

\subsection{Impact on Culture}
Videogames, though a relatively new form of entertainment, has already cemented a significant place in modern culture.
Exhibitions of videogame history, including showcases of certain old videogames, have toured museums, indicating that videogames may even be a form of art.
We should strive to ensure the longevity of videogame media for future generations to explore.
If we look to early examples of broadcast media such film, TV, and radio we see many examples of culture which has been lost, possibly forever.
Whenever old recordings are recovered, this is usually met with celebration, showcasing the cultural value in retaining access to old media\cite{bbc-archive-2001}.
Videogames are subject to damage and loss just as any other form of media.
One of the most useful techniques for safeguarding games for cultural preservation is archival of game \glspl{asset} and gaming systems.
But when publishers make choices that result in killing games, it makes it impossible to preserve them without turning to legally questionable means\footnote{
    Such as bypassing DRM or using emulators which may violate copyright laws.
}.

\lm{It is different if a product was designed by an independent artist with a view to destroy it at a certain point, with the destruction forming part of the artistic work (ref banksy).
But this is never the case for works published commercially.
Any refs to back this up,is this point worth making?}

\subsection{Impact on Broader Consumer Rights}
The issue of preserving videogame playability is strongly related to adjacent areas of modern consumer rights as our world becomes more intertwined with technology.

Right to repair legislation is increasingly gaining attention from lawmakers, with recent announcements made in the USA\cite{us-repair-2024}, the EU\cite{eu-repair-2024}, and the UK\cite{uk-repair-2021}.
Some of these right to repair proposals impose new requirements on the manufacturers or sellers of devices, but this is considered a good trade-off for the wider benefit to society.
An accompaniment to right to repair for physical devices could include the right to repair or modify software components.
This may be necessary where software upgrades are needed in order to ensure that devices can work in the long term,
and a natural consequence of this would be that players of games that they own should be allowed to repair or modify game \glspl{asset} to ensure that they remain playable.
Just as physical device manufacturers will need to engage with right to repair, game publishers could be subject to right to repair rulings as well.

Ownership rights are increasingly coming under threat with the rise of technical measures such as \gls{drm}.
Decisions about how consumers can interact with and the state of ownership of  digital files they have purchased has an impact on the videogames,
but also adjacent creative industries such as ebooks, audiobooks, TV and Film.
\lm{needs a stronger final sentence}

The impact on digital entertainment products may seem frivolous,
but the decisions made here have an impact in setting precedent with regards to safety critical systems.
Globally farmers are seeing an impact to their ability to work and maintain economic stability as a result of internet connected technology in appliances such as tractors, impacting our food production security.
John Deere, a farm equipment manufacturer, has placed restrictions on how farmers can interact with and repair their technology,
resulting in financial burdens from call out fees to repair equipment to loss of crops due to wait times for repairs that previously could have been done directly\cite{farm-repair-2024}.
DRM,resulting in an impact on repairability, was recently found in use in trains in Poland, resulting in  a threat to transport infrastructure.
We will also face similar issues with medical security as more medical devices use internet connected and digital components\cite{internet-bodies-2019}.

\section{Impact on individuals and groups}
We consider some of the impact that could be faced by certain groups or individuals as a result of our proposals.
\lm{Some government actions needs to consider the impact on marginalised groups, or other impacts.
find if there is a correct term for this. ``impact assessment''? Does that need to be something more formal?}

\subsection{Minority demographic groups}
We do not envision any negative impact on minority demographic groups as a result of these proposals.

\subsection{Socio-economic challenges}
Buying a game can be a large investment\footnote{
    Consoles and PCs are needed to play videogames so purchasing these requires an investment of hundreds of pounds,
    and games must be purchased separately.
    Games from large publishers can sell for £60 or more new,
    and with the rise in online connected games and digital only sales,
    the ability to re-sell and buy used games at a discounted rate is disappearing as an option for consumers.
}, so for players facing economic difficulty, during a cost of living crisis, these proposals may help.
Assurances that games would be playable in the long term would mean that any purchases made will last longer and so could be more justifiable.

Games may be used by some people as a way of connecting to others and combating loneliness or social isolation,
and well designed multiplayer games can help with this\cite{design-friends-2018}.
Continued support in some fashion for online multiplayer games could ensure that such players can continue to connect through games even post-support.
As well as relationships, games offer benefits to individuals across multiple dimensions of mental health,
including meaning and accomplishment\cite{gaming-wellbeing-2014}.
If players that rely on videogames to assist with their mental health find them unplayable, with no action possible to repair them,
this could have adverse effects on their wellbeing.

\subsection{Videogame Developers}
Our proposals are applicable to videogame publishers.
It is very unlikely to have a negative impact on videogame developers and other workers in the industry, and is far more likely to benefit them.
Employees across the game industry are increasingly under stress from factors including strenuous development practices and management issues\cite{industry-stress-2020}.
To see their work rendered unplayable could worsen this situation.
Many videogame developers have voiced their dissatisfaction\cn with having a game they spent years of their lives working on destroyed by their publisher, being powerless to stop it.
By having laws requiring the game to function, it would help their work and legacy endure.
It is possible a small number of developers could find new requirements problematic if they were unprepared for them,
but we anticipate if our proposals were implemented, there would be a significant lead-in time giving developers time to prepare for the changes.

Smaller studios and independent video game developers do not typically build and distribute games of the type that are vulnerable to destructive shutdowns when support is withdrawn,
so would not be affected by these proposals.


\section{Counter-arguments}
Opponents of the Stop Killing Games campaign may make certain counter arguments which we will discuss here.

\subsection{Status of Legality}
Questions may arise over the legal status of game ownership and the responsibility of publishers to customers.

Publishers may make the case that games are not sold as products to be owned, but are merely licensed.
This is a legal grey area, depending on how each country interprets the agreements made.
In the United States, this is generally the case\cn.
In other countries, the law is not clear at all, since license agreements cannot override national laws.
Furthermore, many existing laws are not written for a scenario where the seller destroys the product sold to the customer after the point of sale,
since this is not something that normally happens in the real world.
The fact that there is so much ambiguity on this practice is part of why we are running this campaign.

\subsubsection{The UK}
Some countries may be closer to recognising this issue.
For example, in the UK, the Consumer Rights Act defines rules for digital content\cite{cra-digital-2015}.
The consumer rights here specify that individuals have the right to request repair for broken digital content,
and that despite what the license agreement may say, any part of an \gls{eula} would be invalid if it prevented consumers from exercising these rights.
However, it is not clear what constitutes ``repairing'' a game when the publisher is the one who broke it at some point after sale.

\lm{Add the remarks from my DMCA response response}

% Some terms still apply, however.
% For example, you are typically only sold your individual copy of the game license for personal use, not the intellectual property rights to the videogame itself.
% The issue of game ownership, and whether publishers have a responsibility to build games with long term playability in mind may be settles in the US but not in many other countries.


\subsection{Impact on Multiplayer Games}
The way end of support is handled may differ greatly between singleplayer games with online connectivity and always-online multiplayer games.
However, we still believe that publishers must ensure long term playability of games after end of support in all cases.

It may be argued that making multiplayer games work without publisher support is impossible.
This is not the case.
The majority of online multiplayer games in the past functioned without any company servers
and connectivity was handled by the customers privately hosting servers themselves and connecting to each other.
Games that were designed this way are all still playable today.
Whether it is practical to build multiplayer games that are playable in the long term, this can vary significantly depending on the development process behind the game.
If a company has designed a game with no thought given towards the possibility of letting users run the game without their support, then yes,
this can be a challenging goal to transition to.
If a game has been designed with proper end of support as a requirement for completion of the project, then this process can be trivial and relatively simple to implement,
as demonstrated by examples of games given in Chapter~\ref{ch:evidence}.
Making existing games playable may be problematic if they were not designed with this in mind,
but with an appropriate framework set out by law, there is no reason it needs to be difficult for games of the future.

Some games with very large player counts, such as MMORPGs, may have unique requirements for online technology like servers.
It may be argued that this precludes offering post-support playability.
while it may never be possible to completely recreate the infrastructure used for the game during support,
limited playability can still be achieved, ensuring that players can at least play their purchased games in some form.
Several MMORPGs that have been shut down have seen ``server emulators'' emerge that are capable of hosting thousands of other players, just on a single user's system.
Not all will be this scalable, however.
For extra demanding videogames that require powerful servers the average user will not have access to,
the game will not be playable on the same scale as when the developer or publisher was hosting it.
That said, that is no excuse for players not to be able to continue playing the game in some form once support ends.
So, if a server could originally support 5000 people, but the end user version can only support 500,
that's still a massive improvement from no one being able to play the game ever again.
While there are many community built server emulators projects for killed games\cite{emulator-list-2022},
these are often subject to legal issues, and there is no safety guarantee of the software provided to players.
An official publisher provided server emulator would offer safety to players.

It may be argued that this is not worth pursuing as it will not be possible to recreate and maintain every single feature post-support.
We understand some features can be impractical for an end user to attain if running a post-support game.% server only an end-user system.
That said, we also see the ability to continue playing the game in some form as a reasonable demand from companies customers have given money to.
There is a large difference between a game missing some features versus being completely unplayable in any form.

Publishers may argue that post-support playability requirements would ban or pre-empt online only games from being developed in the first place.
We are not asking to interfere with any business activity whatsoever while the game was being actively supported.
The regulations we are seeking would only apply when companies decided to end support for games.
At that time, they would need to be converted to have either offline play or private hosting modes.
Until then, companies could continue developing and publishing games any way they see fit.

The status of multiplayer games where players get banned may be raised.
While the game is being supported, we do not ask for any action.
All our measures are focused on what becomes of the game once support ends.
So if disruptive players in an online-only game become banned,
they will not be able to play the game online while regular players may continue playing with active support.
% then they would not be entitled to run the game offline until support officially ended, which could be many years later.
By end of support, any official infrastructure for handling player authentication or banning would be shut down anyway,
and community run or private servers would take responsibility for configure banning as they please.

\subsection{Impact to publishers}
Publishers may argue hat the proposals we make would have an undue impact on the viability of the videogame publishing model.
Here we offer explanations for why this is not the case.
The primary reason that publishers will not see any long term impact is that we are not asking for any changes to the law or handling of games during the existing support period model.

We are also not asking for publishers to support games forever.
We are in favour of publishers ending support for a game whenever they choose.
What we are asking for is that they implement an end-of-life plan to \gls{update} the game
so that it can run on customer systems with no further support from the company being necessary.
We agree it is unrealistic to expect companies to support games indefinitely and do not advocate for that in any way.

The proposals we make would not have any impact on videogame piracy.
Any financial loss that results from piracy takes place when a game is supported and is being sold in stores.
As our requested action is targeting games after support ends, after players have made their purchases, the problem of piracy is out of scope.

Some publishers create free-to-play games where there is no initial purchase.
While free-to-play games are free for users to try, they are supported by \glspl{microtransaction}, which customers spend money on.
When a publisher ends a free-to-play game without providing any recourse to the players, they are effectively robbing those that bought features for the game.
Hence, they should be accountable to making the game playable in some fashion once support ends.
Our proposed regulations would have no impact on non-commercial games that are 100\% free, however.

Our proposals focus on ensuring that games are playable post-support, and this may raise issues of licensing and intellectual property.
We are not asking for any changes to copyright or intellectual property laws.
So there would be no requirement that the publisher to give up any of its intellectual property rights.
We simply want to ensure players who purchased the game can continue running it.
In no way would that involve the publisher forfeit any intellectual property rights.
Some games are sold containing \glspl{asset} which include licensed content, such as music tracks.
While licenses like this can be a problem for the industry, those would only prohibit the company from selling additional copies of the game once their license expires.
They would not prevent existing buyers from continuing to use the game they have already paid for.

In terms of information and technological security, we do not foresee any risk to publishers or players.
In asking for a game to be operable, we are not demanding that all internal code and documentation be released, just a functional copy of the game.
It would be no more of a security risk than selling the game in the first place was.
There would also be no impact for privacy legislation like the GDPR as we are not asking for private data about players or purchasers to be released,
we are only focusing on playability of the game itself.

The question of cost to implement the changes we ask for may be raised.
It is extremely unlikely that this will risk bankrupting or financially burdening any companies.
The costs associated with implementing this requirement can be very small, if not trivial\cn.
Furthermore, it often takes a company with large resources at its disposal to even construct games vulnerable to loss of playability in the first place,
as setting up server infrastructure and licensing DRM software requires a large initial investment.
Small or independent developers with constrained budgets are less likely to be contributing to this problem,
and as noted in Chapter~\ref{ch:evidence}, it may be that smaller companies are better able to offer good end of support for their games.
