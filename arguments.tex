\chapter{Arguments}

\section{Arguments for}
\lm{Present arguments why we should not kill video games}

Analogy to physical ownership, it cant be taken away

When the warranty (support) for a physical good expires, the item that was under warranty does not immediately stop functioning.
With digital goods there is no wear and tear, so when the support period ends, there is no reason that a digital file should have to stop functioning.
The publisher would not be obligated to provide any support for things like malfunction or bugs after the support period ends.
They should not be allowed remove functionality at the end of support.

Loss of art. Consider example of loss of early film, TV.

Impact on other areas, such as right to repair, Internet of Things, Medical technology\cn
Also, the increase in right to repair legislation being adopted could be leveraged to explain why game support should be extended.\cn



\section{Counter-arguments}
\lm{Discuss some of the points that might be made against this, and explain why we should still stop killing games.
Much of these points will come from the SKG website, but they need citations to backup the claims that they make}

\subsection{legal}
are games licensed or sold?
This is a decision that the government must make clear.
``The short answer is this is a large legal grey area, depending on the country.
In the United States, this is generally the case.\cn
In other countries, the law is not clear at all, since license agreements cannot override national laws.
Those laws often consider videogames as goods, which have many consumer protections that apply to them.
So despite what the license agreement may say, in some countries you are indeed sold your copy of the game license.
Some terms still apply, however.
For example, you are typically only sold your individual copy of the game license for personal use, not the intellectual property rights to the videogame itself.''

law settled?
`` It mostly is within the United States, but not in many other countries.
Many existing laws are not written for a scenario where the seller destroys the product sold to the customer after the point of sale,
since this is not something that normally happens in the real world.
The fact that there is so much ambiguity on this practice is part of why we're pursuing so many legal avenues.''


\subsection{Multi-player}
multiplayer only
`` Not at all.
The majority of online multiplayer games in the past functioned without any company servers
and was conducted by the customers privately hosting servers themselves and connecting to each other.
Games that were designed this way are all still playable today.
As to the practicality, this can vary significantly.
If a company has designed a game with no thought given towards the possibility of letting users run the game without their support, then yes,
this can be a challenging goal to transition to.
If a game has been designed with that as an eventual requirement, then this process can be trivial and relatively simple to implement.
Another way to look at this is it could be problematic for some games of today, but there is no reason it needs to be for games of the future.''

large mmmorpg
`` Not at all, however limitations can apply.
Several MMORPGs that have been shut down have seen 'server emulators' emerge that are capable of hosting thousands of other players, just on a single user's system.
Not all will be this scalable, however. For extra demanding videogames that require powerful servers the average user will not have access to,
the game will not be playable on the same scale as when the developer or publisher was hosting it.
That said, that is no excuse for players not to be able to continue playing the game in some form once support ends.
So, if a server could originally support 5000 people, but the end user version can only support 500,
that's still a massive improvement from no one being able to play the game ever again.''

All features
``Not necessarily. We understand some features can be impractical for an end user to attain if running a server only an end-user system.
That said, we also see the ability to continue playing the game in some form as a reasonable demand from companies customers have given money to.
There is a large difference between a game missing some features versus being completely unplayable in any form.''

ban or pre-empt online only
``Not at all. In fact, nothing we are seeking would interfere with any business activity whatsoever while the game was being actively supported.
The regulations we are seeking would only apply when companies decided to end support for games.
At that time, they would need to be converted to have either offline or private hosting modes.
Until then, companies could continue running games any way they see fit.''

banned players
``Not while the game is being supported.
All our measures are focused on what becomes of the game once support ends.
So if disruptive players in an online-only game become banned, but regular players may continue playing with active support,
then they would not be entitled to run the game offline until support officially ended, which could be many years later.\cn''


\subsection{Impact to developers / publishers}
only applies after end of support

Forever support
`` No, we are not asking that at all. We are in favor of publishers ending support for a game whenever they choose.
What we are asking for is that they implement an end-of-life plan to modify or patch the game
so that it can run on customer systems with no further support from the company being necessary.
We agree it is unrealistic to expect companies to support games indefinitely and do not advocate for that in any way.''

What impact would this have on videogame piracy?
Piracy takes place when a game is supported and is being sold. As this action is targeting games after support ends, this is out of scope.

free-to-play
``While free-to-play games are free for users to try, they are supported by microtransactions, which customers spend money on.
When a publisher ends a free-to-play game without providing any recourse to the players, they are effectively robbing those that bought features for the game.
Hence, they should be accountable to making the game playable in some fashion once support ends.
Our proposed regulations would have no impact on non-commercial games that are 100\% free, however.''

intellectual property
``No, we would not require the company to give up any of its intellectual property rights, simply to allow players who purchased the game to continue running it.
In no way would that involve the publisher forfeit any intellectual property rights.''

perpetual licensing of assets
``No. While those can be a problem for the industry, those would only prohibit the company from selling additional copies of the game once their license expires.
They would not prevent existing buyers from continuing to use the game they have already paid for''

security risk
``Not at all.
In asking for a game to be operable, we're not demanding all internal code and documentation, just a functional copy of the game.
It would be no more of a security risk than selling the game in the first place was.''

general harm
``It is very unlikely, and is far more likely to benefit them.
Many videogame developers have voiced their dissatisfaction\cn with having a game they spent years of their lives working on destroyed by their publisher, being powerless to stop it.
By having laws requiring the game to function, it would help their work and legacy endure.
It is possible a small number of developers could find new requirements problematic if they were unprepared for them, but we anticipate if implemented, there would be a significant lead-in time giving developers time to prepare for the changes.''

cost / bankruptcy
`` It is extremely unlikely. The costs associated with implementing this requirement can be very small, if not trivial.\cn
Furthermore, it often takes a company with large resources at its disposal to even construct games of this nature in the first place.\cn
Small developers with constrained budgets are less likely to be contributing to this problem.''

\section{Impact on groups}

\lm{Some government actions needs to consider the impact on marginalised groups, or other impacts}

Those on low incomes might be affected by the loss of games more than those on higher incomes, as the cost of the game (TODO cite figures) would have more of an impact.
