\chapter{Introduction}

\section{Stop Killing Games}
\lm{Summarise the content of this report}
This report aims to gather and present the evidence and argumentation for why governments should take action to preserve game ownership.
It includes a description of what killing a game involves,
examples of where games have been killed and counter-examples of games that have seen support ended in a responsible fashion,
a summary of the arguments made,
and a series of recommendations for how to protect game ownership.
We also include a glossary of to explain technical or videogame specific terms.

\lm{Include a focused explanation of what this action is asking for, and what it is not asking for}

https://www.stopkillinggames.com

\lm{Find evidence to demonstrate:}
* What are video games
* Why this matters
* Who plays games
* How much the games industry is worth (and how much customers spend on it)

Videogames are a relatively new form of entertainment.
\lm{Is there debate on whether they are art or product?}
Yet the videogames industry has quickly risen to be one of the most profitable businesses

\subsection{What is ``killing a game''?}
Killing a game is when, at some point after a player has purchased a game, the game as purchased becomes no longer playable because the publisher has decided to end support.

\subsection{Goals}
The goal of this campaign - how it will succeed
``If companies face penalties for destroying copies of games they have sold, this is very likely to start curbing this behavior.
If a company is forced to allow customers to retain their games in even one country, implementing those fixes worldwide becomes a trivial issue for them.
So, if destroying a game you paid for became illegal in France, companies that patched the game would likely apply the same patch to the games worldwide.
An analogy to this process is how the ACCC in Australia forced Valve to offer refunds on Steam, so Valve ended up offering them to people worldwide as a result.
''

\lm{This paragraph might be too focused at getting action from petition signers etc.
Rewriting it to focus on key goals from a legislative perspective might be better.
Also, want to avoid giving the sense of over reach from any one specific government as that might turn them away.}

Settle the legal position of game ownership: is publishers killing games legally permitted or not.
If it is not permitted, lay out a framework in which games are to be developed, published and sold in the future.
