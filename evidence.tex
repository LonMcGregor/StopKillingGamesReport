\chapter{Dead Games}

\section{Intro}
\lm{Present the concept of dead games}
There are many ways a game can ``die''.
For our purpose, we will consider cases where a game has lost playable features at the end of the support period, which it had at the start.
Games could be partially functional, for example retaining a single player version of the game but losing access to features within it like online leaderboards\cn.
Games could have major loss of features, for example a combination single and multiplayer game completely losing the multiplayer aspect but retaining the single player version.
Separate components of a game could die at different rates, for example a game could still be playable, but with additional online content delivered via DLC no longer playable.
Or a game could be completely dead, with no recourse for the player to interact with any of it.

\lm{Big question: What about during the servicing period if a feature is added and then removed?
Minecraft lets you download and choose from almost all of the public builds of the game, if thereis a particular feature you want to play with.
A game like overwatch is constantly tweaking gameplay features throughout and you can only play the latest version.
At end of life, which version should the player have access to}

\lm{Question: Delivery. This campaign is not targetting delivery / download of games files.
Are we making the assumption that backing these up is the responsibility of the player.
And if so, how should we mention it with regards to copy-protection systems?
What about consoles where it may not be easy to back up game files?}

\lm{Summarise the differences between killed games, at risk games, etc.}

\section{Continued Playability Case Studies}
We do not expect publishers to keep support for games forever.
We also accept that in the transition to end of support, some features of games may be lost.
But partial loss of features does not mean a game needs to become completely unplayable.
In this section we will observe some case studies of games where the end of support was handled well and feature loss was minimised.

\subsection{Gran Tourismo Sport --- Retaining Single Player}
`Gran Tourismo Sport' (Polyphony digital, published by Sony Interactive Entertainment) released for the Playstation 4 is a racing motorsports game.
It was published in 2017 and end of support took place in 2024.
The game sold almost 13 Million copies~\cite{gtsport-sales-2023} worldwide.
The game carried a combination of online and non-online gameplay features, including single player mode and competitive multiplayer.
When the game's support was ended, the publisher issued a statement~\cite{gtsport-eos-2023} describing how end of support would function,
including noting where features would remain and would be lost.
This notice was published 3 months ahead of the final end of support date.
\lm{Did they advise the date further in advance of this?}

Gameplay features that would be retained:
\begin{itemize}
    \item Singleplayer mode
    \item Individual player game progression, unlocks, save files
    \item Content included in the base game
    \item Content included in DLC, if it was purchased before end of support
\end{itemize}

Gameplay features that would be lost after end of support:
\begin{itemize}
    \item Online play, including online multiplayer
    \item Customised `liveries' applied to vehicles
    \item Tropies for online play
\end{itemize}

\paragraph*{Discussion}
Offering a good lead time is important so that players know in advance when support will end.

That players can continue to play a majority of the core content of the game is a good post-support outcome.
The loss of online multiplayer means a large function of the game has been lost, which is not good, as a core part of the game involved competitive racing.
\lm{Local multiplayer? Leaderboards? Time tracks? Daily challenges?}
The game was patched to ensure that all of the content in the base game and any purchased DLCs would be retained post-support, which is a good outcome.
The removal of tropies, awards for online play, is an unavoidable consequence of the loss of online support, however this does not impact the core gameplay.
The removal of player-customised appearance settings for content (such as car `liveries') is unfortunate, but this is a cosmetic change which does not impact the core gameplay.

Overall, the main game can still be played in some form, which is perferrable to losing all access to the game,
and as long as players retain the console they purchased the game on, they will be able to continue playing it beyond end of support.
The loss of online play is unfortunate, and had the game been designed with an option to use an alternative server or direct player-to-player connection this could have been avoided.
The loss of player customisations need not have happened if the game were designed such that these customisations could be stored locally.


\subsection{Knockout City --- Offering Private Servers}
`Knockout City' (Velan Studios) was a multi-platform multiplayer game similar to dodgeball published in 2021.
The game is an online multiplayer, free-to-play title, with purchases an in-game currency with which players could purchase cosmetic cusotmisations.
The game had 12 million players\cite{knockout-stats}, and in 2022 the developers Velan Studios took over publishing from Electronic Arts.
In February 2023, the studio announced plans to end support for the game in June 2023~\cite{knockout-eos-2023}.

This game was an online-only game, meaning that after end of support, without further intervention, the game would be left in an unplayable state.
However, the developers decided to release a private server copy of the game for PC players~\cite{knockout-private}.
This new PC version would allow players to connect directly to each other rather than requiring going through the publisher's servers.
It also included all of the previously paid-for cosmetic items and game levels.
Private server versions were not made available for console platforms such as Nintendo Switch, Xbox or Playstation.
The new PC version was made available for free.

\paragraph*{Discussion}
The final outcome for this game is that it will be playable forever, as long as players have a working copy of the game and server.
This is an illustrative example that even modern massively multiplayer competitive games can be built using the old model of distributing server software to players,
and that this is the best model for ensuring long term game playability.
Allowing players of the new PC private server version to retain access to all content means that there would be no loss of purchase to the player.
\lm{How to word that this demonstrates there is no loss of sales / profit, given the game is already shut down and cosmetics were just given away free to everyone}
The original developer took charge of publishing from Electronic Arts before the game's end of support, which may have allowed them more control over how to handle the shutdown.
It offered 4 months of lead time in it's end of service announcement.

That console players lost access to the game is unfortunate, this is a limitation of the distribution model on consoles where players have less freedom to configure games.
However as it is freely available for PC, players could pick the game back up at any time, if they use a PC.

\lm{Caller's bane / Scrolls is an example similar to Knockout. Server release.}

\subsection{Mega Man X DiVE --- Re-releasing an offline version}
``Mega Man X DiVE'' (Capcom) was an action game released for mobile platforms in 2020.
It launched with single player, co-operative and competitive multiplayer.
Over its lifetime it has TODO players\cn,
\lm{and the game offered micro-transactions to players???}
In June 2023, the publisher announced end of support for the game, which would end in september 2023\cite{megaman-eos-2023}.
In August 2023, an offline version of the game was announced~\cite{megaman-offline}.

The offline version of the game was offered for sale on multiple platforms, including PC and mobile.
This offline version includes the single player part of the game, and instead of microtransactions, it offers additional content as DLC.

\paragraph*{Discussion}
This is an illustration of a case where a game has been ported from online to offline, essentially creating a new game with features from the previous one.
The shift from a free to play online game to a paid offline game demonstrates that it is economically viable for a publisher to build both types of game.
\lm{As the original game was free, but with microtransactions, it is questionable whether players who have paid for in-game content should have to pay again to }
The loss of the multiplayer portion of the game is unfortunate, but with the new version of the game at least players can continue to experience the main single player gameplay.

\subsection{Duelyst --- Releasing Source Code}
``Duelyst'' (Counterplay Games) was a card strategy game self-published and later published by Bandai Namco.
The game includes online TODO\lm{Finish investigation}

\paragraph*{Discussion}
Releasing the full source code for a game indicates that when a game reaches end of support and a publisher wishes to no longer invest in a game,
there is no harm that comes from releasing the core functionality of the game for for players to use freely.


\section{Killed Games}

\lm{Highlight some large examples of killed games. Include figures for player counts, pricing, dates, evidence and commentary}

\lm{Include a summary table of killed games, and the reason}
Table~\ref{tab:killedgames} shows a summary of high profile games that have been killed, drawing from information crowdsourced online~\cite{dead-games-list-2016}.

\begin{landscape}
\begin{table}[htbp]
    \centering
    \caption{List of killed games}%
    \label{tab:killedgames}
    \begin{tabular*}{1\textwidth}{lllllll}
        Game & Publisher & Launch Date & Support End & Lifetime & Player Count & Note \\ \toprule
        Example Game & Example Publisher & 2010-01-31 & 2012-01-31 & 2 Years & 200K & Online server removed \\
    \end{tabular*}
\end{table}
\end{landscape}

\begin{landscape}
    \begin{table}[htbp]
        \centering
        \caption{List of games with good end of support action}%
        \label{tab:savedgames}
        \begin{tabular*}{1\textwidth}{lllllll}
            Game & Publisher & Launch Date & Support End & Lifetime & Player Count & Note \\ \toprule
            Example Game & Example Publisher & 2010-01-31 & 2012-01-31 & 2 Years & 200K & Server exe offered \\
        \end{tabular*}
    \end{table}
\end{landscape}
