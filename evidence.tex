\chapter{Dead Games}

\section{Intro}
\lm{Present the concept of dead games}

\lm{Summarise the differences between killed games, at risk games, etc.}

\section{Killed Games}

\lm{Highlight some large examples of killed games. Include figures for player counts, pricing, dates, evidence and commentary}

\lm{Include a summary table of killed games, and the reason}

\begin{landscape}
\begin{table}[htbp]
    \centering
    \caption{List of killed games}%
    \label{tab:killedgames}
    \begin{tabular*}{1\textwidth}{lllllll}
        Game & Publisher & Launch Date & Support End & Lifetime & Player Count & Note \\ \toprule
        Example Game & Example Publisher & 2010-01-31 & 2012-01-31 & 2 Years & 200K & Online server removed \\
    \end{tabular*}
\end{table}
\end{landscape}

\section{Good Examples}
\lm{The FAQ of the site includes a number of examples where games were shut down well, but it is not elaborated on how this is the case.
Include these examples here and explain why these are good}

`Gran Turismo Sport' published by Sony

`Knockout City' published by Electronic Arts

`Mega Man X DiVE' published by Capcom

`Scrolls / Caller\'s Bane' published by Mojang AB

`Duelyst' published by Bandai Namco Entertainment

\begin{landscape}
    \begin{table}[htbp]
        \centering
        \caption{List of games with good end of support action}%
        \label{tab:savedgames}
        \begin{tabular*}{1\textwidth}{lllllll}
            Game & Publisher & Launch Date & Support End & Lifetime & Player Count & Note \\ \toprule
            Example Game & Example Publisher & 2010-01-31 & 2012-01-31 & 2 Years & 200K & Server exe offered \\
        \end{tabular*}
    \end{table}
\end{landscape}
